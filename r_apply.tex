% Options for packages loaded elsewhere
\PassOptionsToPackage{unicode}{hyperref}
\PassOptionsToPackage{hyphens}{url}
%
\documentclass[
]{article}
\usepackage{amsmath,amssymb}
\usepackage{iftex}
\ifPDFTeX
  \usepackage[T1]{fontenc}
  \usepackage[utf8]{inputenc}
  \usepackage{textcomp} % provide euro and other symbols
\else % if luatex or xetex
  \usepackage{unicode-math} % this also loads fontspec
  \defaultfontfeatures{Scale=MatchLowercase}
  \defaultfontfeatures[\rmfamily]{Ligatures=TeX,Scale=1}
\fi
\usepackage{lmodern}
\ifPDFTeX\else
  % xetex/luatex font selection
\fi
% Use upquote if available, for straight quotes in verbatim environments
\IfFileExists{upquote.sty}{\usepackage{upquote}}{}
\IfFileExists{microtype.sty}{% use microtype if available
  \usepackage[]{microtype}
  \UseMicrotypeSet[protrusion]{basicmath} % disable protrusion for tt fonts
}{}
\makeatletter
\@ifundefined{KOMAClassName}{% if non-KOMA class
  \IfFileExists{parskip.sty}{%
    \usepackage{parskip}
  }{% else
    \setlength{\parindent}{0pt}
    \setlength{\parskip}{6pt plus 2pt minus 1pt}}
}{% if KOMA class
  \KOMAoptions{parskip=half}}
\makeatother
\usepackage{xcolor}
\usepackage[margin=1in]{geometry}
\usepackage{color}
\usepackage{fancyvrb}
\newcommand{\VerbBar}{|}
\newcommand{\VERB}{\Verb[commandchars=\\\{\}]}
\DefineVerbatimEnvironment{Highlighting}{Verbatim}{commandchars=\\\{\}}
% Add ',fontsize=\small' for more characters per line
\usepackage{framed}
\definecolor{shadecolor}{RGB}{248,248,248}
\newenvironment{Shaded}{\begin{snugshade}}{\end{snugshade}}
\newcommand{\AlertTok}[1]{\textcolor[rgb]{0.94,0.16,0.16}{#1}}
\newcommand{\AnnotationTok}[1]{\textcolor[rgb]{0.56,0.35,0.01}{\textbf{\textit{#1}}}}
\newcommand{\AttributeTok}[1]{\textcolor[rgb]{0.13,0.29,0.53}{#1}}
\newcommand{\BaseNTok}[1]{\textcolor[rgb]{0.00,0.00,0.81}{#1}}
\newcommand{\BuiltInTok}[1]{#1}
\newcommand{\CharTok}[1]{\textcolor[rgb]{0.31,0.60,0.02}{#1}}
\newcommand{\CommentTok}[1]{\textcolor[rgb]{0.56,0.35,0.01}{\textit{#1}}}
\newcommand{\CommentVarTok}[1]{\textcolor[rgb]{0.56,0.35,0.01}{\textbf{\textit{#1}}}}
\newcommand{\ConstantTok}[1]{\textcolor[rgb]{0.56,0.35,0.01}{#1}}
\newcommand{\ControlFlowTok}[1]{\textcolor[rgb]{0.13,0.29,0.53}{\textbf{#1}}}
\newcommand{\DataTypeTok}[1]{\textcolor[rgb]{0.13,0.29,0.53}{#1}}
\newcommand{\DecValTok}[1]{\textcolor[rgb]{0.00,0.00,0.81}{#1}}
\newcommand{\DocumentationTok}[1]{\textcolor[rgb]{0.56,0.35,0.01}{\textbf{\textit{#1}}}}
\newcommand{\ErrorTok}[1]{\textcolor[rgb]{0.64,0.00,0.00}{\textbf{#1}}}
\newcommand{\ExtensionTok}[1]{#1}
\newcommand{\FloatTok}[1]{\textcolor[rgb]{0.00,0.00,0.81}{#1}}
\newcommand{\FunctionTok}[1]{\textcolor[rgb]{0.13,0.29,0.53}{\textbf{#1}}}
\newcommand{\ImportTok}[1]{#1}
\newcommand{\InformationTok}[1]{\textcolor[rgb]{0.56,0.35,0.01}{\textbf{\textit{#1}}}}
\newcommand{\KeywordTok}[1]{\textcolor[rgb]{0.13,0.29,0.53}{\textbf{#1}}}
\newcommand{\NormalTok}[1]{#1}
\newcommand{\OperatorTok}[1]{\textcolor[rgb]{0.81,0.36,0.00}{\textbf{#1}}}
\newcommand{\OtherTok}[1]{\textcolor[rgb]{0.56,0.35,0.01}{#1}}
\newcommand{\PreprocessorTok}[1]{\textcolor[rgb]{0.56,0.35,0.01}{\textit{#1}}}
\newcommand{\RegionMarkerTok}[1]{#1}
\newcommand{\SpecialCharTok}[1]{\textcolor[rgb]{0.81,0.36,0.00}{\textbf{#1}}}
\newcommand{\SpecialStringTok}[1]{\textcolor[rgb]{0.31,0.60,0.02}{#1}}
\newcommand{\StringTok}[1]{\textcolor[rgb]{0.31,0.60,0.02}{#1}}
\newcommand{\VariableTok}[1]{\textcolor[rgb]{0.00,0.00,0.00}{#1}}
\newcommand{\VerbatimStringTok}[1]{\textcolor[rgb]{0.31,0.60,0.02}{#1}}
\newcommand{\WarningTok}[1]{\textcolor[rgb]{0.56,0.35,0.01}{\textbf{\textit{#1}}}}
\usepackage{graphicx}
\makeatletter
\def\maxwidth{\ifdim\Gin@nat@width>\linewidth\linewidth\else\Gin@nat@width\fi}
\def\maxheight{\ifdim\Gin@nat@height>\textheight\textheight\else\Gin@nat@height\fi}
\makeatother
% Scale images if necessary, so that they will not overflow the page
% margins by default, and it is still possible to overwrite the defaults
% using explicit options in \includegraphics[width, height, ...]{}
\setkeys{Gin}{width=\maxwidth,height=\maxheight,keepaspectratio}
% Set default figure placement to htbp
\makeatletter
\def\fps@figure{htbp}
\makeatother
\setlength{\emergencystretch}{3em} % prevent overfull lines
\providecommand{\tightlist}{%
  \setlength{\itemsep}{0pt}\setlength{\parskip}{0pt}}
\setcounter{secnumdepth}{-\maxdimen} % remove section numbering
\ifLuaTeX
  \usepackage{selnolig}  % disable illegal ligatures
\fi
\IfFileExists{bookmark.sty}{\usepackage{bookmark}}{\usepackage{hyperref}}
\IfFileExists{xurl.sty}{\usepackage{xurl}}{} % add URL line breaks if available
\urlstyle{same}
\hypersetup{
  pdftitle={Apply},
  pdfauthor={Masoud Faridi},
  hidelinks,
  pdfcreator={LaTeX via pandoc}}

\title{Apply}
\author{Masoud Faridi}
\date{2024-03-04}

\begin{document}
\maketitle

\hypertarget{apply-function-in-r}{%
\subsection{apply() function in R}\label{apply-function-in-r}}

When you want to apply a function to the rows or columns of a matrix
(and higher-dimensional analogues); not generally advisable for data
frames as it will coerce to a matrix first.

\hypertarget{apply}{%
\subsubsection{apply()}\label{apply}}

\hypertarget{description}{%
\paragraph{Description}\label{description}}

Returns a vector or array or list of values obtained by applying a
function to margins of an array or matrix.

\hypertarget{usage}{%
\paragraph{Usage}\label{usage}}

apply(X, MARGIN, FUN, \ldots, simplify = TRUE)

\hypertarget{arguments}{%
\paragraph{Arguments}\label{arguments}}

\begin{itemize}
\item
  X\\
  an array, including a matrix.
\item
  MARGIN\\
  a vector giving the subscripts which the function will be applied
  over. E.g., for a matrix 1 indicates rows, 2 indicates columns, c(1,
  2) indicates rows and columns. Where X has named dimnames, it can be a
  character vector selecting dimension names.
\item
  FUN\\
  the function to be applied: see `Details'. In the case of functions
  like +, \%*\%, etc., the function name must be backquoted or quoted.
\item
  \ldots{}\\
  optional arguments to FUN.
\item
  simplify\\
  a logical indicating whether results should be simplified if possible.
\item
  na.rm = TRUE\textbar FALSE
\end{itemize}

\hypertarget{examples}{%
\paragraph{Examples}\label{examples}}

Let

\[
\text{mat}=\left(
\begin{array}{ccc}
1 & 2 & 3 \\
4 & 5 & 6\\
7 & 8 & 9\\
10 & 11 & 12\\
\end{array}
\right).
\]

\begin{Shaded}
\begin{Highlighting}[]
\NormalTok{mat}\OtherTok{\textless{}{-}}\FunctionTok{matrix}\NormalTok{(}\DecValTok{1}\SpecialCharTok{:}\DecValTok{12}\NormalTok{,}\AttributeTok{byrow=}\NormalTok{T,}\AttributeTok{ncol=}\DecValTok{3}\NormalTok{,}\AttributeTok{nrow=}\DecValTok{4}\NormalTok{)}
\FunctionTok{apply}\NormalTok{(}\AttributeTok{X=}\NormalTok{mat, }\AttributeTok{MARGIN=}\DecValTok{1}\NormalTok{, }\AttributeTok{FUN=}\NormalTok{sum)}
\end{Highlighting}
\end{Shaded}

\begin{verbatim}
## [1]  6 15 24 33
\end{verbatim}

\begin{Shaded}
\begin{Highlighting}[]
\FunctionTok{apply}\NormalTok{(}\AttributeTok{X=}\NormalTok{mat, }\AttributeTok{MARGIN=}\DecValTok{2}\NormalTok{, }\AttributeTok{FUN=}\NormalTok{sum)}
\end{Highlighting}
\end{Shaded}

\begin{verbatim}
## [1] 22 26 30
\end{verbatim}

\begin{Shaded}
\begin{Highlighting}[]
\CommentTok{\#Last, if you apply the function to each cell:}
\FunctionTok{apply}\NormalTok{(}\AttributeTok{X=}\NormalTok{mat, }\AttributeTok{MARGIN=}\FunctionTok{c}\NormalTok{(}\DecValTok{1}\NormalTok{,}\DecValTok{2}\NormalTok{), }\AttributeTok{FUN=}\NormalTok{sum)}
\end{Highlighting}
\end{Shaded}

\begin{verbatim}
##      [,1] [,2] [,3]
## [1,]    1    2    3
## [2,]    4    5    6
## [3,]    7    8    9
## [4,]   10   11   12
\end{verbatim}

\begin{Shaded}
\begin{Highlighting}[]
\FunctionTok{apply}\NormalTok{(}\AttributeTok{X=}\NormalTok{mat, }\AttributeTok{MARGIN=}\FunctionTok{c}\NormalTok{(}\DecValTok{1}\NormalTok{,}\DecValTok{2}\NormalTok{), }\AttributeTok{FUN=}\ControlFlowTok{function}\NormalTok{(x) x}\SpecialCharTok{\^{}}\DecValTok{2}\NormalTok{)}
\end{Highlighting}
\end{Shaded}

\begin{verbatim}
##      [,1] [,2] [,3]
## [1,]    1    4    9
## [2,]   16   25   36
## [3,]   49   64   81
## [4,]  100  121  144
\end{verbatim}

\begin{Shaded}
\begin{Highlighting}[]
\CommentTok{\#If you set MARGIN = c(2, 1) instead of c(1, 2) the output will be the same matrix \#but transposed.}
\FunctionTok{apply}\NormalTok{(}\AttributeTok{X=}\NormalTok{mat, }\AttributeTok{MARGIN=}\FunctionTok{c}\NormalTok{(}\DecValTok{2}\NormalTok{,}\DecValTok{1}\NormalTok{), }\AttributeTok{FUN=}\ControlFlowTok{function}\NormalTok{(x) x}\SpecialCharTok{\^{}}\DecValTok{2}\NormalTok{)}
\end{Highlighting}
\end{Shaded}

\begin{verbatim}
##      [,1] [,2] [,3] [,4]
## [1,]    1   16   49  100
## [2,]    4   25   64  121
## [3,]    9   36   81  144
\end{verbatim}

You can set the MARGIN argument to c(1, 2) or, equivalently, to 1:2 to
apply the function to each value of the matrix. Note that, in this case,
the elements of the output are the elements of the matrix itself, as it
is calculating the sum of each individual cell.

If you set MARGIN = c(2, 1) instead of c(1, 2) the output will be the
same matrix but transposed.

\begin{Shaded}
\begin{Highlighting}[]
\FunctionTok{apply}\NormalTok{(mat, }\DecValTok{2}\NormalTok{, range)   }\CommentTok{\# Range (min and max values) by column}
\end{Highlighting}
\end{Shaded}

\begin{verbatim}
##      [,1] [,2] [,3]
## [1,]    1    2    3
## [2,]   10   11   12
\end{verbatim}

\begin{Shaded}
\begin{Highlighting}[]
\FunctionTok{apply}\NormalTok{(mat, }\DecValTok{1}\NormalTok{, summary) }\CommentTok{\# Summary for each row}
\end{Highlighting}
\end{Shaded}

\begin{verbatim}
##         [,1] [,2] [,3] [,4]
## Min.     1.0  4.0  7.0 10.0
## 1st Qu.  1.5  4.5  7.5 10.5
## Median   2.0  5.0  8.0 11.0
## Mean     2.0  5.0  8.0 11.0
## 3rd Qu.  2.5  5.5  8.5 11.5
## Max.     3.0  6.0  9.0 12.0
\end{verbatim}

\begin{Shaded}
\begin{Highlighting}[]
\FunctionTok{apply}\NormalTok{(mat, }\DecValTok{2}\NormalTok{, summary) }\CommentTok{\# Summary for each column}
\end{Highlighting}
\end{Shaded}

\begin{verbatim}
##          [,1]  [,2]  [,3]
## Min.     1.00  2.00  3.00
## 1st Qu.  3.25  4.25  5.25
## Median   5.50  6.50  7.50
## Mean     5.50  6.50  7.50
## 3rd Qu.  7.75  8.75  9.75
## Max.    10.00 11.00 12.00
\end{verbatim}

\begin{Shaded}
\begin{Highlighting}[]
\NormalTok{mat}\OtherTok{\textless{}{-}}\FunctionTok{matrix}\NormalTok{(}\DecValTok{1}\SpecialCharTok{:}\DecValTok{9}\NormalTok{,}\AttributeTok{byrow=}\NormalTok{T,}\AttributeTok{ncol=}\DecValTok{3}\NormalTok{)}
\FunctionTok{apply}\NormalTok{(}\AttributeTok{X=}\NormalTok{mat, }\AttributeTok{MARGIN=}\DecValTok{1}\NormalTok{, }\AttributeTok{FUN=}\NormalTok{sum)}\SpecialCharTok{==}\FunctionTok{rowSums}\NormalTok{(mat)}
\end{Highlighting}
\end{Shaded}

\begin{verbatim}
## [1] TRUE TRUE TRUE
\end{verbatim}

\begin{Shaded}
\begin{Highlighting}[]
\FunctionTok{apply}\NormalTok{(}\AttributeTok{X=}\NormalTok{mat, }\AttributeTok{MARGIN=}\DecValTok{2}\NormalTok{, }\AttributeTok{FUN=}\NormalTok{sum)}\SpecialCharTok{==}\FunctionTok{colSums}\NormalTok{(mat)}
\end{Highlighting}
\end{Shaded}

\begin{verbatim}
## [1] TRUE TRUE TRUE
\end{verbatim}

\begin{Shaded}
\begin{Highlighting}[]
\CommentTok{\# Return the product of each of the rows}
\FunctionTok{apply}\NormalTok{(}\AttributeTok{X=}\NormalTok{mat,}\AttributeTok{MARGIN=}\DecValTok{1}\NormalTok{,}\AttributeTok{FUN=}\NormalTok{prod)}
\end{Highlighting}
\end{Shaded}

\begin{verbatim}
## [1]   6 120 504
\end{verbatim}

\begin{Shaded}
\begin{Highlighting}[]
\CommentTok{\# Return the sum of each of the columns}
\FunctionTok{apply}\NormalTok{(}\AttributeTok{X=}\NormalTok{mat,}\AttributeTok{MARGIN=}\DecValTok{2}\NormalTok{,}\AttributeTok{FUN=}\NormalTok{min)}
\end{Highlighting}
\end{Shaded}

\begin{verbatim}
## [1] 1 2 3
\end{verbatim}

\hypertarget{applying-a-custom-function}{%
\paragraph{Applying a custom
function}\label{applying-a-custom-function}}

\begin{itemize}
\tightlist
\item
  تابع یک بردار را می گیرد و یک عدد را بر می گرداند
\end{itemize}

در این حالت خروجی یک بردار است

\begin{Shaded}
\begin{Highlighting}[]
\CommentTok{\# Return a new matrix whose entries are those of \textquotesingle{}m\textquotesingle{} modulo 10}
\FunctionTok{apply}\NormalTok{(mat,}\DecValTok{1}\NormalTok{,}\ControlFlowTok{function}\NormalTok{(x) }\FunctionTok{sum}\NormalTok{(x))}\SpecialCharTok{==} \FunctionTok{rowSums}\NormalTok{(mat)}
\end{Highlighting}
\end{Shaded}

\begin{verbatim}
## [1] TRUE TRUE TRUE
\end{verbatim}

\begin{Shaded}
\begin{Highlighting}[]
\CommentTok{\# Return a new matrix whose entries are those of \textquotesingle{}m\textquotesingle{} modulo 10}
\FunctionTok{apply}\NormalTok{(mat,}\DecValTok{2}\NormalTok{,}\ControlFlowTok{function}\NormalTok{(x) }\FunctionTok{mean}\NormalTok{(x))}\SpecialCharTok{==} \FunctionTok{colMeans}\NormalTok{(mat)}
\end{Highlighting}
\end{Shaded}

\begin{verbatim}
## [1] TRUE TRUE TRUE
\end{verbatim}

\begin{itemize}
\tightlist
\item
  تایع یک بردار را می گیرد و یک بردار را بر می گرداند
\end{itemize}

در این حالت خروجی یک ماتریس است و بدرد مواردی می خورد که بخواهیم روی هر
عنصر یک تابع خاص را اعمال کنیم.

در این حالت چه روی سطر چه روی ستون تابع اعمال شود تفاوتی ندارد

\begin{Shaded}
\begin{Highlighting}[]
\CommentTok{\# Return a new matrix whose entries are those of \textquotesingle{}m\textquotesingle{} modulo 10}
\FunctionTok{apply}\NormalTok{(mat,}\DecValTok{1}\NormalTok{,}\ControlFlowTok{function}\NormalTok{(x) x}\SpecialCharTok{\%\%}\DecValTok{2}\NormalTok{)}
\end{Highlighting}
\end{Shaded}

\begin{verbatim}
##      [,1] [,2] [,3]
## [1,]    1    0    1
## [2,]    0    1    0
## [3,]    1    0    1
\end{verbatim}

\begin{Shaded}
\begin{Highlighting}[]
\CommentTok{\# Return a new matrix whose entries are those of \textquotesingle{}m\textquotesingle{} modulo 10}
\FunctionTok{apply}\NormalTok{(mat,}\DecValTok{2}\NormalTok{,}\ControlFlowTok{function}\NormalTok{(x) x}\SpecialCharTok{\%\%}\DecValTok{2}\NormalTok{)}
\end{Highlighting}
\end{Shaded}

\begin{verbatim}
##      [,1] [,2] [,3]
## [1,]    1    0    1
## [2,]    0    1    0
## [3,]    1    0    1
\end{verbatim}

\hypertarget{passing-arguments-to-iterated-function-through-apply}{%
\paragraph{Passing arguments to iterated function through
apply}\label{passing-arguments-to-iterated-function-through-apply}}

Note the \ldots{} in the function definition:

\begin{Shaded}
\begin{Highlighting}[]
\FunctionTok{args}\NormalTok{(apply)}
\end{Highlighting}
\end{Shaded}

\begin{verbatim}
## function (X, MARGIN, FUN, ..., simplify = TRUE) 
## NULL
\end{verbatim}

and the corresponding entry in the documentation:

\ldots: optional arguments to `FUN'.

\begin{Shaded}
\begin{Highlighting}[]
\NormalTok{f1 }\OtherTok{\textless{}{-}} \ControlFlowTok{function}\NormalTok{(x,v2)\{}
\NormalTok{  x}\SpecialCharTok{+}\NormalTok{v2}
\NormalTok{\}}
\FunctionTok{apply}\NormalTok{(}\AttributeTok{X=}\NormalTok{mat,}\AttributeTok{MARGIN =} \FunctionTok{c}\NormalTok{(}\DecValTok{1}\NormalTok{,}\DecValTok{2}\NormalTok{), }\AttributeTok{FUN =}\NormalTok{ f1,}\AttributeTok{v2=}\DecValTok{1}\NormalTok{)}
\end{Highlighting}
\end{Shaded}

\begin{verbatim}
##      [,1] [,2] [,3]
## [1,]    2    3    4
## [2,]    5    6    7
## [3,]    8    9   10
\end{verbatim}

\hypertarget{lapply}{%
\subsubsection{lapply()}\label{lapply}}

lapply() function is useful for performing operations on list objects
and returns a list object of same length of original set. lappy()
returns a list of the similar length as input list object, each element
of which is the result of applying FUN to the corresponding element of
list. Lapply in R takes list, vector or data frame as input and gives
output in list.

\begin{Shaded}
\begin{Highlighting}[]
\CommentTok{\#Example}
\CommentTok{\#Sort the list alphabetically:}
\NormalTok{movies }\OtherTok{\textless{}{-}} \FunctionTok{c}\NormalTok{(}\StringTok{"SPYDERMAN"}\NormalTok{,}\StringTok{"BATMAN"}\NormalTok{,}\StringTok{"VERTIGO"}\NormalTok{,}\StringTok{"CHINATOWN"}\NormalTok{)}
\NormalTok{movies\_lower }\OtherTok{\textless{}{-}}\FunctionTok{lapply}\NormalTok{(movies, tolower)}
\FunctionTok{str}\NormalTok{(movies\_lower)}
\end{Highlighting}
\end{Shaded}

\begin{verbatim}
## List of 4
##  $ : chr "spyderman"
##  $ : chr "batman"
##  $ : chr "vertigo"
##  $ : chr "chinatown"
\end{verbatim}

\begin{Shaded}
\begin{Highlighting}[]
\NormalTok{ls1}\OtherTok{\textless{}{-}}\FunctionTok{list}\NormalTok{(}\AttributeTok{a=}\DecValTok{1}\SpecialCharTok{:}\DecValTok{3}\NormalTok{,}\AttributeTok{b=}\FunctionTok{rep}\NormalTok{(}\DecValTok{1}\SpecialCharTok{:}\DecValTok{2}\NormalTok{,}\AttributeTok{each=}\DecValTok{4}\NormalTok{),}\AttributeTok{c=}\DecValTok{1}\SpecialCharTok{:}\DecValTok{9}\NormalTok{)}
\FunctionTok{lapply}\NormalTok{(ls1,length)}
\end{Highlighting}
\end{Shaded}

\begin{verbatim}
## $a
## [1] 3
## 
## $b
## [1] 8
## 
## $c
## [1] 9
\end{verbatim}

\begin{Shaded}
\begin{Highlighting}[]
\NormalTok{ls1}\OtherTok{\textless{}{-}}\FunctionTok{list}\NormalTok{(}\AttributeTok{a=}\DecValTok{1}\SpecialCharTok{:}\DecValTok{3}\NormalTok{,}\AttributeTok{b=}\DecValTok{1}\SpecialCharTok{:}\DecValTok{4}\NormalTok{,}\AttributeTok{c=}\DecValTok{1}\SpecialCharTok{:}\DecValTok{5}\NormalTok{)}
\FunctionTok{lapply}\NormalTok{(ls1,cumsum)}
\end{Highlighting}
\end{Shaded}

\begin{verbatim}
## $a
## [1] 1 3 6
## 
## $b
## [1]  1  3  6 10
## 
## $c
## [1]  1  3  6 10 15
\end{verbatim}

\hypertarget{description-1}{%
\paragraph{Description}\label{description-1}}

lapply returns a list of the same length as X, each element of which is
the result of applying FUN to the corresponding element of X.

\hypertarget{arguments-1}{%
\paragraph{Arguments}\label{arguments-1}}

\begin{itemize}
\item
  X a vector (atomic or list) or an expression object. Other objects
  (including classed objects) will be coerced by base::as.list.
\item
  FUN\\
  the function to be applied to each element of X: see `Details'. In the
  case of functions like +, \%*\%, the function name must be backquoted
  or quoted.
\item
  \ldots{}\\
  optional arguments to FUN.
\end{itemize}

\hypertarget{example-1}{%
\paragraph{Example 1}\label{example-1}}

\begin{Shaded}
\begin{Highlighting}[]
\NormalTok{x }\OtherTok{\textless{}{-}} \FunctionTok{list}\NormalTok{(}\AttributeTok{a =} \DecValTok{1}\SpecialCharTok{:}\DecValTok{10}\NormalTok{, }\AttributeTok{beta =} \FunctionTok{exp}\NormalTok{(}\SpecialCharTok{{-}}\DecValTok{3}\SpecialCharTok{:}\DecValTok{3}\NormalTok{), }\AttributeTok{logic =} \FunctionTok{c}\NormalTok{(}\ConstantTok{TRUE}\NormalTok{,}\ConstantTok{FALSE}\NormalTok{,}\ConstantTok{FALSE}\NormalTok{,}\ConstantTok{TRUE}\NormalTok{))}
\FunctionTok{lapply}\NormalTok{(x, mean)}
\end{Highlighting}
\end{Shaded}

\begin{verbatim}
## $a
## [1] 5.5
## 
## $beta
## [1] 4.535125
## 
## $logic
## [1] 0.5
\end{verbatim}

\hypertarget{applying-a-custom-function-1}{%
\paragraph{Applying a custom
function}\label{applying-a-custom-function-1}}

\begin{itemize}
\tightlist
\item
  تابع یک بردار را می گیرد و یک بردار را بر می گرداند در این حالت فرمت و
  بعد خروجی با ورودی یکسان است
\end{itemize}

\begin{Shaded}
\begin{Highlighting}[]
\NormalTok{x }\OtherTok{\textless{}{-}} \FunctionTok{list}\NormalTok{(}\AttributeTok{a =} \DecValTok{1}\SpecialCharTok{:}\DecValTok{10}\NormalTok{, }\AttributeTok{b =} \DecValTok{5}\SpecialCharTok{:}\DecValTok{1}\NormalTok{)}
\FunctionTok{lapply}\NormalTok{(x, }\ControlFlowTok{function}\NormalTok{(x) x}\SpecialCharTok{\^{}}\DecValTok{2}\NormalTok{)}
\end{Highlighting}
\end{Shaded}

\begin{verbatim}
## $a
##  [1]   1   4   9  16  25  36  49  64  81 100
## 
## $b
## [1] 25 16  9  4  1
\end{verbatim}

\begin{itemize}
\tightlist
\item
  تابع یک بردار را می گیرد و یک عدد را بر می گرداند در این حالت به
  اندازه طول لیست، خروجی داریم. یعنی به ازای هر بردار در لیست مورد نظر
  یک خروجی خواهیم داشت
\end{itemize}

\begin{Shaded}
\begin{Highlighting}[]
\NormalTok{x }\OtherTok{\textless{}{-}} \FunctionTok{list}\NormalTok{(}\AttributeTok{a =} \DecValTok{1}\SpecialCharTok{:}\DecValTok{10}\NormalTok{, }\AttributeTok{b =} \DecValTok{5}\SpecialCharTok{:}\DecValTok{1}\NormalTok{)}
\FunctionTok{lapply}\NormalTok{(x, }\ControlFlowTok{function}\NormalTok{(x) }\FunctionTok{sum}\NormalTok{(x))}
\end{Highlighting}
\end{Shaded}

\begin{verbatim}
## $a
## [1] 55
## 
## $b
## [1] 15
\end{verbatim}

\begin{Shaded}
\begin{Highlighting}[]
\NormalTok{x }\OtherTok{\textless{}{-}} \FunctionTok{list}\NormalTok{(}\AttributeTok{a =} \DecValTok{1}\SpecialCharTok{:}\DecValTok{3}\NormalTok{, }\AttributeTok{b =} \DecValTok{1}\SpecialCharTok{:}\DecValTok{4}\NormalTok{)}
\FunctionTok{lapply}\NormalTok{(x, }\ControlFlowTok{function}\NormalTok{(x) (}\FunctionTok{min}\NormalTok{(x)}\SpecialCharTok{+}\FunctionTok{max}\NormalTok{(x))}\SpecialCharTok{/}\DecValTok{2}\NormalTok{)}
\end{Highlighting}
\end{Shaded}

\begin{verbatim}
## $a
## [1] 2
## 
## $b
## [1] 2.5
\end{verbatim}

\hypertarget{passing-arguments-to-iterated-function-through-apply-1}{%
\paragraph{Passing arguments to iterated function through
apply}\label{passing-arguments-to-iterated-function-through-apply-1}}

Note the \ldots{} in the function definition:

\begin{Shaded}
\begin{Highlighting}[]
\NormalTok{x }\OtherTok{\textless{}{-}} \FunctionTok{list}\NormalTok{(}\AttributeTok{a =} \DecValTok{1}\SpecialCharTok{:}\DecValTok{10}\NormalTok{, }\AttributeTok{b =} \DecValTok{5}\SpecialCharTok{:}\DecValTok{1}\NormalTok{)}
\FunctionTok{lapply}\NormalTok{(x, }\ControlFlowTok{function}\NormalTok{(x,vr) vr}\SpecialCharTok{+}\FunctionTok{sum}\NormalTok{(x),}\AttributeTok{vr=}\DecValTok{7}\NormalTok{)}
\end{Highlighting}
\end{Shaded}

\begin{verbatim}
## $a
## [1] 62
## 
## $b
## [1] 22
\end{verbatim}

\begin{Shaded}
\begin{Highlighting}[]
\NormalTok{x }\OtherTok{\textless{}{-}} \FunctionTok{list}\NormalTok{(}\AttributeTok{a =} \DecValTok{1}\SpecialCharTok{:}\DecValTok{3}\NormalTok{, }\AttributeTok{b =} \DecValTok{4}\SpecialCharTok{:}\DecValTok{1}\NormalTok{)}
\FunctionTok{lapply}\NormalTok{(x, }\ControlFlowTok{function}\NormalTok{(x,pow) x}\SpecialCharTok{\^{}}\NormalTok{pow,}\AttributeTok{pow=}\DecValTok{2}\NormalTok{)}
\end{Highlighting}
\end{Shaded}

\begin{verbatim}
## $a
## [1] 1 4 9
## 
## $b
## [1] 16  9  4  1
\end{verbatim}

quantile(x, \ldots)

quantile(x, probs = seq(0, 1, 0.25), na.rm = FALSE, names = TRUE, type =
7, digits = 7, \ldots)

\begin{Shaded}
\begin{Highlighting}[]
\NormalTok{x }\OtherTok{\textless{}{-}} \FunctionTok{list}\NormalTok{(}\AttributeTok{a =} \DecValTok{1}\SpecialCharTok{:}\DecValTok{10}\NormalTok{, }\AttributeTok{beta =} \FunctionTok{exp}\NormalTok{(}\SpecialCharTok{{-}}\DecValTok{3}\SpecialCharTok{:}\DecValTok{3}\NormalTok{), }\AttributeTok{logic =} \FunctionTok{c}\NormalTok{(}\ConstantTok{TRUE}\NormalTok{,}\ConstantTok{FALSE}\NormalTok{,}\ConstantTok{FALSE}\NormalTok{,}\ConstantTok{TRUE}\NormalTok{))}
\FunctionTok{lapply}\NormalTok{(x, quantile, }\AttributeTok{probs =}\NormalTok{ (}\DecValTok{1}\SpecialCharTok{:}\DecValTok{3}\NormalTok{)}\SpecialCharTok{/}\DecValTok{4}\NormalTok{)}
\end{Highlighting}
\end{Shaded}

\begin{verbatim}
## $a
##  25%  50%  75% 
## 3.25 5.50 7.75 
## 
## $beta
##       25%       50%       75% 
## 0.2516074 1.0000000 5.0536690 
## 
## $logic
## 25% 50% 75% 
## 0.0 0.5 1.0
\end{verbatim}

\hypertarget{lapply-vs-for-loop}{%
\paragraph{lapply vs for loop}\label{lapply-vs-for-loop}}

The lapply function can be used to avoid for loops, which are known to
be slow in R when not used properly. Consider that you want to return a
list containing the third power of the even numbers of a vector and the
the fourth power of the odd numbers of that vector. In that case you
could type:

\begin{Shaded}
\begin{Highlighting}[]
\CommentTok{\# Empty list with 5 elements}
\NormalTok{x }\OtherTok{\textless{}{-}} \FunctionTok{vector}\NormalTok{(}\StringTok{"list"}\NormalTok{, }\DecValTok{5}\NormalTok{)}

\CommentTok{\# Vector}
\NormalTok{vec }\OtherTok{\textless{}{-}} \DecValTok{1}\SpecialCharTok{:}\DecValTok{5}

\ControlFlowTok{for}\NormalTok{(i }\ControlFlowTok{in}\NormalTok{ vec) \{}
    \ControlFlowTok{if}\NormalTok{(i }\SpecialCharTok{\%\%} \DecValTok{2} \SpecialCharTok{==} \DecValTok{0}\NormalTok{) \{ }\CommentTok{\# Check if the element \textquotesingle{}i\textquotesingle{} is even or odd}
\NormalTok{        x[[i]] }\OtherTok{\textless{}{-}}\NormalTok{ i }\SpecialCharTok{\^{}} \DecValTok{3}
\NormalTok{    \} }\ControlFlowTok{else}\NormalTok{ \{}
\NormalTok{        x[[i]] }\OtherTok{\textless{}{-}}\NormalTok{ i }\SpecialCharTok{\^{}} \DecValTok{4}
\NormalTok{    \}}
\NormalTok{\}}
\NormalTok{x}
\end{Highlighting}
\end{Shaded}

\begin{verbatim}
## [[1]]
## [1] 1
## 
## [[2]]
## [1] 8
## 
## [[3]]
## [1] 81
## 
## [[4]]
## [1] 64
## 
## [[5]]
## [1] 625
\end{verbatim}

An alternative is to use the lappy function as follows:

\begin{Shaded}
\begin{Highlighting}[]
\NormalTok{vec }\OtherTok{\textless{}{-}} \DecValTok{1}\SpecialCharTok{:}\DecValTok{5}
\NormalTok{fun }\OtherTok{\textless{}{-}} \ControlFlowTok{function}\NormalTok{(i) \{}
   \ControlFlowTok{if}\NormalTok{(i }\SpecialCharTok{\%\%} \DecValTok{2} \SpecialCharTok{==} \DecValTok{0}\NormalTok{) \{}
\NormalTok{        i }\SpecialCharTok{\^{}} \DecValTok{3}
\NormalTok{   \} }\ControlFlowTok{else}\NormalTok{ \{}
\NormalTok{        i }\SpecialCharTok{\^{}} \DecValTok{4}
\NormalTok{    \}}
\NormalTok{\}}

\FunctionTok{lapply}\NormalTok{(vec, fun)}
\end{Highlighting}
\end{Shaded}

\begin{verbatim}
## [[1]]
## [1] 1
## 
## [[2]]
## [1] 8
## 
## [[3]]
## [1] 81
## 
## [[4]]
## [1] 64
## 
## [[5]]
## [1] 625
\end{verbatim}

You will only be able to use the lapply function instead of a for loop
if you want to return a list of the same length as the vector or list
you want to iterate with.

\hypertarget{using-lapply-on-certain-columns-of-an-r-data-frame}{%
\paragraph{Using lapply on certain columns of an R data
frame}\label{using-lapply-on-certain-columns-of-an-r-data-frame}}

Consider that you have a data frame and you want to multiply the
elements of the first column by one, the elements of the second by two
and so on.

On the one hand, for all columns you could write:

\begin{Shaded}
\begin{Highlighting}[]
\NormalTok{df }\OtherTok{\textless{}{-}} \FunctionTok{data.frame}\NormalTok{(}\AttributeTok{x =} \FunctionTok{c}\NormalTok{(}\DecValTok{6}\NormalTok{, }\DecValTok{2}\NormalTok{), }\AttributeTok{y =} \FunctionTok{c}\NormalTok{(}\DecValTok{3}\NormalTok{, }\DecValTok{6}\NormalTok{), }\AttributeTok{z =} \FunctionTok{c}\NormalTok{(}\DecValTok{2}\NormalTok{, }\DecValTok{3}\NormalTok{))}
\NormalTok{df}
\end{Highlighting}
\end{Shaded}

\begin{verbatim}
##   x y z
## 1 6 3 2
## 2 2 6 3
\end{verbatim}

\begin{Shaded}
\begin{Highlighting}[]
\CommentTok{\# Function applied to all columns}
\FunctionTok{lapply}\NormalTok{(}\DecValTok{1}\SpecialCharTok{:}\FunctionTok{ncol}\NormalTok{(df), }\ControlFlowTok{function}\NormalTok{(i) df[, i] }\SpecialCharTok{*}\NormalTok{ i)}
\end{Highlighting}
\end{Shaded}

\begin{verbatim}
## [[1]]
## [1] 6 2
## 
## [[2]]
## [1]  6 12
## 
## [[3]]
## [1] 6 9
\end{verbatim}

\begin{Shaded}
\begin{Highlighting}[]
\CommentTok{\#or lapply(1:ncol(df), function(i,mat) mat[, i] * i,mat=df)}
\end{Highlighting}
\end{Shaded}

On the other hand, If you want to use the lapply function to certain
columns of the data frame you could type:

\begin{Shaded}
\begin{Highlighting}[]
\CommentTok{\# Function applied to the first and third columns}
\FunctionTok{lapply}\NormalTok{(}\FunctionTok{c}\NormalTok{(}\DecValTok{1}\NormalTok{, }\DecValTok{3}\NormalTok{), }\ControlFlowTok{function}\NormalTok{(i) df[, i] }\SpecialCharTok{*}\NormalTok{ i)}
\end{Highlighting}
\end{Shaded}

\begin{verbatim}
## [[1]]
## [1] 6 2
## 
## [[2]]
## [1] 6 9
\end{verbatim}

\hypertarget{nested-lapply-functions}{%
\paragraph{Nested lapply functions}\label{nested-lapply-functions}}

If needed, you can nest multiply lapply functions. Consider that you
want to iterate over the columns and rows of a data frame and apply a
function to each cell. For that purpose, and supposing that you want to
multiply each cell by four, you could type something like the following:

\begin{Shaded}
\begin{Highlighting}[]
\NormalTok{df }\OtherTok{\textless{}{-}} \FunctionTok{data.frame}\NormalTok{(}\AttributeTok{x =} \FunctionTok{c}\NormalTok{(}\DecValTok{6}\NormalTok{, }\DecValTok{2}\NormalTok{), }\AttributeTok{y =} \FunctionTok{c}\NormalTok{(}\DecValTok{3}\NormalTok{, }\DecValTok{6}\NormalTok{))}

\CommentTok{\# Empty list}
\NormalTok{res }\OtherTok{\textless{}{-}} \FunctionTok{vector}\NormalTok{(}\StringTok{"list"}\NormalTok{, }\DecValTok{2}\NormalTok{)}

\ControlFlowTok{for}\NormalTok{(i }\ControlFlowTok{in} \DecValTok{1}\SpecialCharTok{:}\FunctionTok{ncol}\NormalTok{(df)) \{}
    \ControlFlowTok{for}\NormalTok{ (j }\ControlFlowTok{in} \DecValTok{1}\SpecialCharTok{:}\FunctionTok{nrow}\NormalTok{(df)) \{}
\NormalTok{        res[[j]][i] }\OtherTok{\textless{}{-}}\NormalTok{ df[j, i] }\SpecialCharTok{*} \DecValTok{4}
\NormalTok{    \}}
\NormalTok{\}}

\NormalTok{res}
\end{Highlighting}
\end{Shaded}

\begin{verbatim}
## [[1]]
## [1] 24 12
## 
## [[2]]
## [1]  8 24
\end{verbatim}

You can get the same values nesting two lapply functions, applying a
lapply inside the FUN argument of the first:

\begin{Shaded}
\begin{Highlighting}[]
\FunctionTok{lapply}\NormalTok{(}\DecValTok{1}\SpecialCharTok{:}\FunctionTok{ncol}\NormalTok{(df), }\ControlFlowTok{function}\NormalTok{(i) \{}
       \FunctionTok{unlist}\NormalTok{(}\FunctionTok{lapply}\NormalTok{(}\DecValTok{1}\SpecialCharTok{:}\FunctionTok{nrow}\NormalTok{(df), }\ControlFlowTok{function}\NormalTok{(j) \{}
\NormalTok{              df[j, i] }\SpecialCharTok{*} \DecValTok{4}
\NormalTok{       \}))}
\NormalTok{\})}
\end{Highlighting}
\end{Shaded}

\begin{verbatim}
## [[1]]
## [1] 24  8
## 
## [[2]]
## [1] 12 24
\end{verbatim}

\begin{Shaded}
\begin{Highlighting}[]
\NormalTok{m}\OtherTok{\textless{}{-}}\FunctionTok{matrix}\NormalTok{(}\DecValTok{0}\NormalTok{,}\DecValTok{2}\NormalTok{,}\DecValTok{3}\NormalTok{)}
\ControlFlowTok{for}\NormalTok{(i }\ControlFlowTok{in} \DecValTok{1}\SpecialCharTok{:}\DecValTok{2}\NormalTok{)\{ }\ControlFlowTok{for}\NormalTok{(j }\ControlFlowTok{in} \DecValTok{1}\SpecialCharTok{:}\DecValTok{3}\NormalTok{) m[i,j]}\OtherTok{\textless{}{-}}\NormalTok{i}\SpecialCharTok{+}\NormalTok{j\}}
\NormalTok{m}
\end{Highlighting}
\end{Shaded}

\begin{verbatim}
##      [,1] [,2] [,3]
## [1,]    2    3    4
## [2,]    3    4    5
\end{verbatim}

\begin{Shaded}
\begin{Highlighting}[]
\FunctionTok{lapply}\NormalTok{(}\DecValTok{1}\SpecialCharTok{:}\DecValTok{2}\NormalTok{, }\ControlFlowTok{function}\NormalTok{(i) \{}\FunctionTok{lapply}\NormalTok{(}\DecValTok{1}\SpecialCharTok{:}\DecValTok{3}\NormalTok{,}\ControlFlowTok{function}\NormalTok{(j) i}\SpecialCharTok{+}\NormalTok{j)\})}
\end{Highlighting}
\end{Shaded}

\begin{verbatim}
## [[1]]
## [[1]][[1]]
## [1] 2
## 
## [[1]][[2]]
## [1] 3
## 
## [[1]][[3]]
## [1] 4
## 
## 
## [[2]]
## [[2]][[1]]
## [1] 3
## 
## [[2]][[2]]
## [1] 4
## 
## [[2]][[3]]
## [1] 5
\end{verbatim}

The lapply and sapply functions are very similar, as the first is a
wrapper of the second. The main difference between the functions is that
lapply returns a list instead of an array. However, if you set simplify
= FALSE to the sapply function both will return a list.

\begin{Shaded}
\begin{Highlighting}[]
\FunctionTok{sapply}\NormalTok{(}\DecValTok{1}\SpecialCharTok{:}\DecValTok{2}\NormalTok{, }\ControlFlowTok{function}\NormalTok{(i) \{}\FunctionTok{sapply}\NormalTok{(}\DecValTok{1}\SpecialCharTok{:}\DecValTok{3}\NormalTok{,}\ControlFlowTok{function}\NormalTok{(j) i}\SpecialCharTok{+}\NormalTok{j)\})}
\end{Highlighting}
\end{Shaded}

\begin{verbatim}
##      [,1] [,2]
## [1,]    2    3
## [2,]    3    4
## [3,]    4    5
\end{verbatim}

\hypertarget{sapply}{%
\subsubsection{sapply()}\label{sapply}}

\begin{Shaded}
\begin{Highlighting}[]
\CommentTok{\#Example}
\CommentTok{\#Sort the list alphabetically:}
\end{Highlighting}
\end{Shaded}

\hypertarget{vapply}{%
\subsubsection{vapply()}\label{vapply}}

\begin{Shaded}
\begin{Highlighting}[]
\CommentTok{\#Example}
\CommentTok{\#Sort the list alphabetically:}
\end{Highlighting}
\end{Shaded}

\hypertarget{tapply}{%
\subsubsection{tapply()}\label{tapply}}

\begin{Shaded}
\begin{Highlighting}[]
\CommentTok{\#Example}
\CommentTok{\#Sort the list alphabetically:}
\end{Highlighting}
\end{Shaded}

\hypertarget{mapply}{%
\subsubsection{mapply()}\label{mapply}}

\begin{Shaded}
\begin{Highlighting}[]
\CommentTok{\#Example}
\CommentTok{\#Sort the list alphabetically:}
\end{Highlighting}
\end{Shaded}

\hypertarget{by}{%
\subsubsection{by()}\label{by}}

\begin{Shaded}
\begin{Highlighting}[]
\CommentTok{\#Example}
\CommentTok{\#Sort the list alphabetically:}
\end{Highlighting}
\end{Shaded}

Map \#\#\# Map()

\begin{Shaded}
\begin{Highlighting}[]
\CommentTok{\#Example}
\CommentTok{\#Sort the list alphabetically:}
\end{Highlighting}
\end{Shaded}

\hypertarget{replicate}{%
\subsubsection{replicate()}\label{replicate}}

\begin{Shaded}
\begin{Highlighting}[]
\CommentTok{\#Example}
\CommentTok{\#Sort the list alphabetically:}
\end{Highlighting}
\end{Shaded}

\hypertarget{contact-us}{%
\subsection{Contact us}\label{contact-us}}

Contact me at
\href{mailto:masoudfaridi@modares.ac.ir}{\nolinkurl{masoudfaridi@modares.ac.ir}}
or
\href{mailto:masoud1faridi@gmail.com}{\nolinkurl{masoud1faridi@gmail.com}}

\end{document}
